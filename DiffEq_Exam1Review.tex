\documentclass{ximera}
%% handout
%% nohints
%% space
%% newpage
%% numbers

%\input{preamble.tex} %% we can turn off input when making a master document
\usepackage{fullpage}

\begin{document}
\title[Math 240:]{Exam 1 Practice}

\begin{enumerate}
	
\item

	\begin{enumerate}[label=\theenumii)]
	
	\item Check directly whether $y = x^2$ is a solution to the differential equation
	$$y^{\prime\prime} + y^\prime = 0.$$
	
	\vfill
	
	Answer: When $y = x^2$, $y^{\prime\prime} + y = \answer{2x+2}$. So it \wordChoice[is]{is not} a solution to the differential equation.
	
	\item Check directly whether $y = x^2$ is a solution to the initial value problem
	$$y^{\prime\prime} + 2 = 4,     y(1) = 1,    y^\prime(1) = 2.$$
	
	\vfill
	
	Answer: When $y = x^2$, $y^{\prime\prime} + 2 = \answer{4}$, $y(1) = \answer{1}$, and $y^\prime(1) = \answer{2}$. So it \wordChoice[is not]{is} a solution to the initial value problem.
	
	\item Check directly whether y = 17 is a solution to the differential equation
	$$y^{(5)} + 18y^{(4)} + 3y^{(3)} -14y^{\prime\prime} + 190y^\prime = 0.$$
	
	\vfill
	
	Answer: When $y = 17$, $y^{(5)} + 18y^{(4)} + 3y^{(3)} -14y^{\prime\prime} + 190y^\prime =  \answer{0}$. So it \wordChoice[is not]{is} a solution to the initial value problem.

	\end{enumerate}

\pagebreak




\end{enumerate}



2a. If you solve a third-order differential equation, how many constants will there be in the solution?










2b. Solve y(3) = 10.











2c. Solve y(3) = y’.





3a. Solve the differential equation y(3) = y. 




















3b. Solve the differential equation y(3) + 2y’’ - 29y’ - 30y = 0.




4a. Solve the differential equation y’ = 2y + 7.





















4b. Solve the differential equation y’ = 2y + 7 using a completely different method.
















5a. Solve the differential equation 12xy + 13x = dy/dx.




















5b. Solve the differential equation x y2 - y y’ + x y4 = 2 y3 y’.


6. Draw three isoclines: one for y’ = 1/4, y’ = 1, and y’ = 4 for the following equation.
	y’ = x2 + y2	 Then use the isoclines to sketch a solution starting at (0, 1).






















7. Give an example of an initial value problem that does not have a solution.



8a. Give two different solutions to the following initial value problem:

	y’ = y, 	y(4) = 0
















8b. Why is it that the Existence and Uniqueness Theorem does not guarantee a unique solution?














9. Find the singular solution to the differential equation y’ = x2y - x2.


10. Solve the following differential equation.

	(D2 + 10D + 26)(D-7)2(D2 - 6D + 34)2D4 y = 0


\end{document}
