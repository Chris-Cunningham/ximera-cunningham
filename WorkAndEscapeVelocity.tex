\documentclass{ximera}
%% handout
%% nohints
%% space
%% newpage
%% numbers

\input{preamble.tex} %% we can turn off input when making a master document

\outcome{Use an improper integral in an application problem.}
\outcome{Integrate a force function to compute work.}
\outcome{Correctly handle units during calculus operations.}

\title{Escape Velocity}
\begin{document}
\begin{abstract}
We start with Newton's law of gravitation and derive the escape velocity of a planet.
\end{abstract}
\maketitle

A simplified version of Newton's Law is as follows. The force of gravity on an object of mass $m$ that is distance $r$ from the center of the earth is given by
$$ F(r) = \frac{k m}{r^2}$$
where $k$ is a constant that is a characteristic of the earth. In the metric system, the units of $k$ are $\frac{m^3}{s^2}$.

\begin{question}
Use the following two facts to find the value of k in $\frac{m^3}{s^2}$ to three significant digits. 
\begin{itemize}
\item A 100-kg object weighs 980 N at the surface of the Earth.
\item The Earth's radius is 6400 km.
\end{itemize}
\begin{solution}
\begin{multiple-choice}
\choice{k = $4.01 \cdot 10^8 \frac{m^3}{s^2}$}
\choice[correct]{k = $4.01 \cdot 10^{14} \frac{m^3}{s^2}$}
\choice{k = $418 \frac{m^3}{s^2}$}
\choice{k = $4.18 \cdot 10^8 \frac{m^3}{s^2}$}
\end{multiple-choice}
\begin{hint}
The two facts from this question can be plugged into the law of gravity.
\end{hint}
\begin{hint}
There is a problem with the units of the Earth's radius. Make sure you convert it to meters!
\end{hint}
Plugging in the values gives $980 N = \frac{k (100 kg)}{(6400000 m)^2}$, and so $k = 4.01 \cdot 10^{14} \frac{m^3}{s^2}$.
\end{solution}
\end{question}

\end{document}
