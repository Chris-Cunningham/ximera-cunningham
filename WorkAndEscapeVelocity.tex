\documentclass{ximera}
%% handout
%% nohints
%% space
%% newpage
%% numbers

\input{preamble.tex} %% we can turn off input when making a master document

\outcome{Use an improper integral in an application problem.}
\outcome{Integrate a force function to compute work.}
\outcome{Correctly handle units during calculus operations.}

\title{Escape Velocity}
\begin{document}
\begin{abstract}
We start with Newton's law of gravitation and derive the escape velocity of a planet.
\end{abstract}
\maketitle

\begin{observation}
\textbf{Newton's Law of Gravitation.} The force of gravity on an object of mass $m$ that is distance $r$ from the center of the earth is given by
$$ F(r) = \frac{G M m}{r^2}$$
where $G$ is the universal gravitational constant and $M$ is the mass of the Earth. In the metric system, the units of $G M$ are $\frac{m^3}{s^2}$.
\end{observation}

\textbf{Finding the value of $GM$}

\begin{question}
Use the following two facts to find the value of $GM$ in $\frac{m^3}{s^2}$ to three significant digits. 
\begin{itemize}
\item A 100-kg object weighs 981 N at the surface of the Earth.
\item The Earth's radius is 6380 km.
\end{itemize}
\begin{solution}
\begin{multiple-choice}
\choice{G M = $3.99 \cdot 10^8 \frac{m^3}{s^2}$}
\choice[correct]{G M = $3.99 \cdot 10^{14} \frac{m^3}{s^2}$}
\choice{G M = $418 \frac{m^3}{s^2}$}
\choice{G M = $4.18 \cdot 10^8 \frac{m^3}{s^2}$}
\end{multiple-choice}
\begin{hint}
The two facts from this question can be plugged into the law of gravity.
\end{hint}
\begin{hint}
There is a problem with the units of the Earth's radius. Make sure you convert it to meters!
\end{hint}
Plugging in the values gives $981 N = \frac{GM (100 kg)}{(6380000 m)^2}$, and so $GM = 3.99 \cdot 10^{14} \frac{m^3}{s^2}$.
\end{solution}
\end{question}

\textbf{Finding the value of $M$}

\begin{question}
The value of $G$ is $6.67 \cdot 10^-{11} \frac{m^3}{kg \cdot s^2}$. Find the mass of the Earth in kilograms. Give 3 significant digits and use scientific notation.
\begin{solution}
\begin{hint}
You found the value of $G M$ in the previous question, so you should now be able to find M by dividing.
\end{hint}
Dividing $G M$ by $G$ yields \answer{$5.99*10^{24}$} kg.
\end{solution}
\end{question}


\end{document}
