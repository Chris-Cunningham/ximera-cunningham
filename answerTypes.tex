\documentclass{ximera}
\title{Question and answer types}
\begin{document}

\begin{abstract}
  This activity explores the answer types and environments supported by ximera.
\end{abstract}

\maketitle

\section{Currently-Supported Environments and Commands (2015-07-28)}

\begin{itemize}
\item Basic Answer Types
  \begin{itemize}
    \item \verb!\answer!
    \item (not documented) \verb!\answer[given]!
    \item \verb!\answer[tolerance=]!
    \item (broken) \verb!\matrixAnswer!    
  \end{itemize}
\item Multiple Choice Answer Types
  \begin{itemize}
    \item \verb!\begin{multipleChoice} ... \end{multipleChoice}!
    \item (broken) \verb!\wordChoice!
  \end{itemize}    
\item Free Response Answers
  \begin{itemize}   
    \item (broken) \verb!\begin{freeResponse} ... \end{freeResponse}!
  \end{itemize}
\item Code Blocks
  \begin{itemize}
    \item (broken) \verb!\begin{code} ... \end{code}!
    \item (broken) \verb!\begin{python} ... \end{python}!
  \end{itemize}
\item Problems and Exercises
  \begin{itemize}
    \item \verb!\begin{problem} ... \end{problem}!
    \item \verb!\begin{question} ... \end{question}!
    \item \verb!\begin{exercise} ... \end{exercise}!
    \item \verb!\begin{exploration} ... \end{exploration}!
    \item \verb!\begin{xarmaBoost} ... \end{xarmaBoost}!
  \end{itemize}
\end{itemize}

\section{Basic Answer Type}

The following code produces a basic answer. In print, this displays the correct answer in an answer box. Online, this displays a blank space for the student to fill in the answer. 

\begin{verbatim}
\begin{question}
$3\times 2 = \answer{6}$
\end{question}
\end{verbatim}

Produces:

\begin{question}
$3\times 2 = \answer{6}$
\end{question}

In addition to numerical answers, we also support elementary functions.

\begin{verbatim}
\begin{question}
  $\frac{\partial}{\partial x} x^2\sin(y) = \answer{2xsin(y)}$
\end{question}
\end{verbatim}

Produces:

\begin{question}
  $\frac{\partial}{\partial x} x^2\sin(y) = \answer{2xsin(y)}$
\end{question}

\begin{remark}
Under the hood, Ximera is parsing the user input, producing a
function, and checking the user input function against ``answer'' at
$100$ different complex numbers, and seeing if the results are
``reasonably'' close to each other.  We compare the complex extensions
of these functions to circumvent domain issues.
\end{remark}

If you would like to allow more tolerance, you can use the \verb!tolerance! option. 

\begin{warning}
Setting the tolerance to 0 will force exact comparison of floating-point numbers, which will likely cause unexpected problems. This is strongly discouraged. 
\end{warning}

\begin{verbatim}
\begin{question}
  Write a number that is within 0.01 of 8: \answer{8}
\end{question}
\begin{question}
  Write a number that is within 5 of 8: \answer[tolerance=5]{8}
\end{question}
\end{verbatim}

Produces:

\begin{question}
  Write a number that is within 0.01 of 8: $\answer{8}$
\end{question}
\begin{question}
  Write a number that is within 5 of 8: $\answer{8}$  \emph{broken as of 7/28/2015:} the ximera class file does not know about tolerance yet.
\end{question}

We also support matrices of expressions in answers.

\begin{verbatim}
\begin{question}
Enter the matrix  \(\begin{bmatrix} x & y \\ xy & z+1 \end{bmatrix}\)
    \begin{matrixAnswer}
	    correctMatrix = [['x','y'],['xy','z+1']]
    \end{matrixAnswer}
\end{question}
\end{verbatim}

\emph{broken as of 7/28/2015:} This should eventually convert to a +/- matrix answer widget.

\begin{question}
Enter the matrix  \(\begin{bmatrix} x & y \\ xy & z+1 \end{bmatrix}\)
\begin{matrixAnswer}
    correctMatrix = [['x','y'],['xy','z+1']]
\end{matrixAnswer}
\end{question}

\begin{remark}
  The plus and minus buttons add and subtract columns or rows.  
\end{remark}


\section{Multiple Choice Answer Type}

Multiple-choice questions need one or more of their answers to be labeled as correct. By default, the options always appear in the same order.

\begin{verbatim}
\begin{question}
Which of the following functions has a graph which is a parabola?
  \begin{multipleChoice}
    \choice[correct]{$y=x^2+3x-3$}
    \choice{$y = \frac{1}{x+2}$}
    \choice{$y=3x+1$}
  \end{multipleChoice}
\end{question}
\end{verbatim}

Produces:

\begin{question}
  Which of the following functions has a graph which is a parabola?
  \begin{multipleChoice}
    \choice[correct]{$y=x^2+3x-3$}
    \choice{$y = \frac{1}{x+2}$}
    \choice{$y=3x+1$}
  \end{multipleChoice}
\end{question}


If you would like the multiple-choice question to occur in-line instead of taking up space, you can use the inline command \verb!wordChoice!.
 
\begin{verbatim}
\begin{question}
  When comparing the functions $f(x) = x^2$ and $g(x) = 2^x$, \wordChoice{\choice{$f(x)$} \choice[correct]{$g(x)$}} grows more quickly for large $x$.
\end{question}
\end{verbatim}

Produces:

\begin{question}
  When comparing the functions $f(x) = x^2$ and $g(x) = 2^x$, \wordChoice{\choice{$f(x)$} \choice[correct]{$g(x)$}} grows more quickly for large $x$.
  \emph{broken as of 7/28/2015:} the web does not understand wordChoice yet at all.
\end{question}

\section{Free-Response Answer Type}

The free response environment gives students access to a \LaTeX\ editor. 

\begin{verbatim}
\begin{question}
  Tell us a little about yourself!
  \begin{freeResponse}
      My name is Professor X. I am 42 years old, originally from Columbus, OH. The most recent 
      math I learned was functional analysis, three years ago. My favorite ice cream is Superman. 
      My favorite function is $f(x) = e^{-x^2}$.
  \end{freeResponse}
\end{question}
\end{verbatim}

Produces:

\begin{question}
  Tell us a little about yourself!
  \begin{freeResponse}
      My name is Professor X. I am 42 years old, originally from Columbus, OH. The most recent 
      math I learned was functional analysis, three years ago. My favorite ice cream is Superman. 
      My favorite function is $f(x) = e^{-x^2}.$
  \end{freeResponse}
\end{question}

\begin{remark}
Clicking on \verb!View model solution! shows the user
whatever you typed in the  \verb!freeResponse! environment.
\end{remark}


\section{Code}

If you are displaying computer code to the student, you can use the \verb!code! or \verb!python! environments.

\begin{verbatim}
\begin{code}
This is in code environment.
    This is indented on the second line.

This is on the fourth line.
\end{code}
\end{verbatim}

Produces:

\begin{code}
This is in code environment.
    This is indented on the second line.

This is on the fourth line.
\end{code}

If you are asking your students to write python code, for example trying to get them to type \verb!|return x**2!, you can use the \verb!python! environment and some validation functions.

\begin{verbatim}
\begin{question}
Model the function $f(x) = x^2$ as a python function. Try using \verb|return x**2|
\begin{python}
  def f(x):
  return #your code here

  def validator():
  return (f(4) == 16) and (f(-5) == 25)
\end{python}
\end{question}
\end{verbatim}

Produces:

\begin{question}
Model the function $f(x) = x^2$ as a python function. Try using \verb|return x**2|
\begin{python}
  def f(x):
  return #your code here

  def validator():
  return (f(4) == 16) and (f(-5) == 25)
\end{python}
\end{question}



\section{Problem Containers}

The environments \verb!problem!, \verb!question!, \verb!exercise!, \verb!exploration!, and \verb!xarmaBoost! are intended to contain various answer types inside. Most of the above examples involve question or problem environments. Nesting these environments will cause a series of questions to appear, requiring you to answer the first before answering the second.

\begin{verbatim}
\begin{question}
  $2+2 = \answer{4}$
  \begin{problem}
  If we add 2 to the result, we get $\answer{6}$
  \end{problem}
\end{question}
\end{verbatim}

Produces:

\begin{question}
  $2+2 = \answer{4}$
  \begin{problem}
  If we add 2 to the result, we get $\answer{6}$
  \end{problem}
\end{question}

\begin{remark} 
The original intent of xarmaBoost was to label questions that were unlikely to be graded, but might just let the student do something interesting. Now, though, any of these question environments can actually contain many answer commands or none.
\end{remark}

\section{Structure}


\verb!example!, \verb!explanation!



\section{Prompt}

\begin{prompt}
This is a prompt.
\end{prompt}



\section{Shuffle}

We need to figure out what the shuffle environment does.

\begin{shuffle}
This is a shuffle environment. What goes here?
\end{shuffle}


\section{Feedback}

We need to figure out what the feedback environment does.

\begin{feedback}
What is this?
\end{feedback}

\section{Dialogue}

Dialogue is working as intended:

\begin{dialogue}
\item[Nathan] What is this?
\item[Hans] This is for showing a conversation.
\end{dialogue}

\section{Theorems}

\begin{theorem}
What is this? (there are other theorems)
\end{theorem}

Maybe we want to list the other theorems?
=======
\begin{lemma}
This is a lemma.
\end{lemma}



\section{Linking}

video, youtube don't work in html
link{with a link here} works
link{} links back to same page online, how does this work?

\video{https://www.youtube.com/watch?v=dQw4w9WgXcQ} %doesn't work in html
\youtube{https://www.youtube.com/watch?v=dQw4w9WgXcQ} %doesn't work in html
\link{https://www.youtube.com/watch?v=dQw4w9WgXcQ} %works
\link{} %links to same page online, does nothing in pdf



\section{Metadata}

There is some metadata here. (prerequisites, outcomes to be figured out later)

\prerequisites{This is a prerequisite} %metadata

\outcome{This is an outcome.} %metadata



\section{Problem Types}


\end{document}
