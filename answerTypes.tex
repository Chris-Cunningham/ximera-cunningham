
\documentclass{ximera}

\title{Question and answer types}
\begin{document}

\begin{abstract}
  This activity explores the answer types and environments supported by ximera, and shows one example of each.
\end{abstract}

\maketitle

\section{Currently-Supported Environments and Commands (2015-07-29)}

\begin{remark} Table of Contents

\begin{itemize}
\item \hyperref[BasicAnswerType]{\textbf{Basic Answer Types}}
  \begin{itemize}
    \item (broken in print) \verb!answer!
    \item (not documented) \verb!answer[given]!
    \item \verb!answer[tolerance=]!
    \item (ugly in print, broken online) \verb!\matrixAnswer!
  \end{itemize}
\item \hyperref[MCAnswerType]{\textbf{Multiple Choice Answer Types}}
  \begin{itemize}
    \item \verb!\begin\{multipleChoice\} ... end\{multipleChoice\}!
    \item (broken online) begin\{selectAll\} ... end\{selectAll\}
    \item (broken online) \textbackslash wordChoice
  \end{itemize}    
\item \hyperref[ProblemContainers]{\textbf{Problems, Exercises, and Hints}}
  \begin{itemize}
    \item begin\{problem\} ... end\{problem\}
    \item begin\{question\} ... end\{question\}
    \item begin\{exercise\} ... end\{exercise\}
    \item begin\{exploration\} ... end\{exploration\}
    \item begin\{xarmaBoost\} ... end\{xarmaBoost\}
    \item (broken in print) begin\{shuffle\} ... end\{shuffle\}
    \item (broken online) begin\{feedback\} ... end\{feedback\}
    \item begin\{hint\} ... end\{hint\}
  \end{itemize}
\item \hyperref[TheoremEnvironments]{\textbf{Theorem Environments}}
  \begin{itemize}
    \item Theorem, Axiom, Conjecture, Corollary, Proposition, Lemma, Claim, Condition, Idea, Definition, Conclusion, Summary
    \item Example, Observation, Fact, Remark, Algorithm, Notation, Criterion
    \item Warning, Paradox
  \end{itemize}
\item \hyperref[EmbeddedContent]{\textbf{Embedded Content}}
  \begin{itemize}
    \item link
    \item youtube
    \item geogebra
    \item (broken in print, broken online) html
  \end{itemize}
\item \hyperref[CodeAnswers]{\textbf{Code Blocks}}
  \begin{itemize}
    \item (ugly online) begin\{code\} ... end\{code\}
    \item (broken in print, broken online) begin\{python\} ... end\{python\}
  \end{itemize}
\item \hyperref[FRAnswerType]{\textbf{Free Response}}
  \begin{itemize}   
    \item (broken online) begin\{freeResponse\} ... end\{freeResponse\}
  \end{itemize}
\item \hyperref[StructuralEnvironments]{\textbf{Miscellaneous Structure}}
  \begin{itemize}
    \item example
    \item explanation
  \end{itemize}
\item \hyperref[Prompts]{\textbf{Prompts}}
  \begin{itemize}
    \item begin\{prompt\} ... end \{prompt\}
  \end{itemize}
\item \hyperref[Dialogue]{\textbf{Dialogues}}
  \begin{itemize}
    \item begin \{dialogue\} ... end\{dialogue\}
  \end{itemize}
\item \hyperref[Metadata]{\textbf{Metadata}}
  \begin{itemize}
    \item (broken in print, broken online) outcome
  \end{itemize}

\end{itemize}

\end{remark}


\section{Basic Answer Types}  
\label{BasicAnswerType}

\begin{example}
The following code produces a basic answer. In print, this displays the correct answer in an answer box. Online, this displays a blank space for the student to fill in the answer. 

\begin{verbatim}
\begin{question}
$3\times 2 = \answer{6}$
\end{question}
\end{verbatim}

Produces:

\begin{question}
$3\times 2 = \answer{6}$
\end{question}
\end{example}

\begin{example}
In addition to numerical answers, we also support elementary functions.

\begin{verbatim}
\begin{question}
  $\frac{\partial}{\partial x} x^2\sin(y) = \answer{2xsin(y)}$
\end{question}
\end{verbatim}

Produces:

\begin{question}
  $\frac{\partial}{\partial x} x^2\sin(y) = \answer{2xsin(y)}$
\end{question}


\begin{remark}
Under the hood, Ximera is parsing the user input, producing a
function, and checking the user input function against ``answer'' at
$100$ different complex numbers, and seeing if the results are
``reasonably'' close to each other.  We compare the complex extensions
of these functions to circumvent domain issues.
\end{remark}
\end{example}


\begin{example} If you would like to allow more tolerance, you can use the \verb!tolerance! option. 

\begin{warning}
Setting the tolerance to 0 will force exact comparison of floating-point numbers, which will likely cause unexpected problems. This is strongly discouraged. 
\end{warning}

\begin{verbatim}
\begin{question}
  Write a number that is within 0.01 of 8: \answer{8}
\end{question}
\begin{question}
  Write a number that is within 5 of 8: \answer[tolerance=5]{8}
\end{question}
\end{verbatim}

Produces:

\begin{question}
  Write a number that is within 0.01 of 8: $\answer{8}$
\end{question}
\begin{question}
  Write a number that is within 5 of 8: $\answer[tolerance=5]{8}$
\end{question}
\end{example}


\begin{example} We also support matrices of expressions in answers.

\begin{verbatim}
\begin{question}
Enter the matrix  \(\begin{bmatrix} x & y \\ xy & z+1 \end{bmatrix}\)
    \begin{matrixAnswer}
	    correctMatrix = [['x','y'],['xy','z+1']]
    \end{matrixAnswer}
\end{question}
\end{verbatim}

\emph{broken as of 7/28/2015:} This should eventually convert to a +/- matrix answer widget.

\begin{question}
Enter the matrix  \(\begin{bmatrix} x & y \\ xy & z+1 \end{bmatrix}\)
\begin{matrixAnswer}
    correctMatrix = [['x','y'],['xy','z+1']]
\end{matrixAnswer}
\end{question}

\begin{remark}
  The plus and minus buttons add and subtract columns or rows.  
\end{remark}
\end{example}

\section{Multiple Choice Answer Types} \label{MCAnswerType}

\begin{example}
Multiple-choice questions need one or more of their answers to be labeled as \verb!correct!. By default, the options always appear in the same order.

\begin{verbatim}
\begin{question}
Which of the following functions has a graph which is a parabola?
  \begin{multipleChoice}
    \choice[correct]{$y=x^2+3x-3$}
    \choice{$y = \frac{1}{x+2}$}
    \choice{$y=3x+1$}
  \end{multipleChoice}
\end{question}
\end{verbatim}

Produces:

\begin{question}
  Which of the following functions has a graph which is a parabola?
  \begin{multipleChoice}
    \choice[correct]{$y=x^2+3x-3$}
    \choice{$y = \frac{1}{x+2}$}
    \choice{$y=3x+1$}
  \end{multipleChoice}
\end{question}

\begin{remark}
If you mark more than one choice as \verb!correct!, then students will be asked to find \emph{one} of the correct answers. If you instead expect students to select \emph{all} correct choices, use the \verb!selectAll! environment below.
\end{remark}

\end{example}

\begin{example}
To allow students to choose more than one option in a multiple choice question, we use the \verb!selectAll! environment. The syntax is identical to the \verb!multipleChoice! environment.

\begin{verbatim}
\begin{question}
Select all the functions below that have a graph which is a parabola.
  \begin{selectAll}
    \choice[correct]{$y=x^2+3x-3$}
    \choice{$y = \frac{1}{x+2}$}
    \choice[correct]{$y=3x^2+1$}
  \end{selectAll}
\end{question}
\end{verbatim}

Produces:

\begin{question}
Select all the functions below that have a graph which is a parabola.
  \begin{selectAll}
    \choice[correct]{$y=x^2+3x-3$}
    \choice{$y = \frac{1}{x+2}$}
    \choice[correct]{$y=3x^2+1$}
  \end{selectAll}
\end{question}

\end{example}

\begin{example}
If you would like the multiple-choice question to occur in-line instead of taking up space, you can use the inline command \verb!wordChoice!.
 
\begin{verbatim}
\begin{question}
  When comparing the functions $f(x) = x^2$ and $g(x) = 2^x$, \wordChoice{\choice{$f(x)$} \choice[correct]{$g(x)$}} grows more quickly for large $x$.
\end{question}
\end{verbatim}

Produces:

\begin{question}
  When comparing the functions $f(x) = x^2$ and $g(x) = 2^x$, \wordChoice{\choice{$f(x)$} \choice[correct]{$g(x)$}} grows more quickly for large $x$.
  \emph{broken as of 7/28/2015:} the web does not understand wordChoice yet at all.
\end{question}
\end{example}

\section{Problems, Exercises, and Hints} \label{ProblemContainers}

\begin{example}
The environments \verb!problem!, \verb!question!, \verb!exercise!, \verb!exploration!, and \verb!xarmaBoost! are intended to contain various answer types inside. Nesting these environments will cause a series of questions to appear, requiring you to answer the first before answering the second. The following code displays nested problems.

\begin{verbatim}
\begin{question}
  $2+2 = \answer{4}$
  \begin{problem}
  If we add 2 to the result, we get $\answer{6}$
  \end{problem}
\end{question}
\end{verbatim}

Produces:

\begin{question}
  $2+2 = \answer{4}$
  \begin{problem}
  If we add 2 to the result, we get $\answer{6}$
  \end{problem}
\end{question}

\begin{remark} 
The original intent of xarmaBoost was to label questions that were unlikely to be graded, but might just let the student do something interesting. Now, though, any of these question environments can actually contain many answer commands or none.
\end{remark}
\end{example}

\begin{example} The \verb!shuffle! environment contains several possible questions; then each student is given one randomly chosen question from the list.

\begin{verbatim}
\begin{shuffle}
\begin{question} Find the x-coordinate of the vertex of the parabola $y = x^2 - 6x + 4$. 
$x = \answer{3}$. \end{question}
\begin{question} Find the x-coordinate of the vertex of the parabola $y = x^2 - 8x + 4$. 
$x = \answer{4}$. \end{question}
\begin{question} Find the x-coordinate of the vertex of the parabola $y = x^2 - 2x + 4$. 
$x = \answer{1}$. \end{question}
\end{shuffle}
\end{verbatim}

Produces something different every time (you should see one of the three questions below):

\begin{shuffle}
\begin{question} Find the x-coordinate of the vertex of the parabola $y = x^2 - 6x + 4$. 
$x = \answer{3}$. \end{question}
\begin{question} Find the x-coordinate of the vertex of the parabola $y = x^2 - 8x + 4$. 
$x = \answer{4}$. \end{question}
\begin{question} Find the x-coordinate of the vertex of the parabola $y = x^2 - 2x + 4$. 
$x = \answer{1}$. \end{question}
\end{shuffle}

\end{example}

\begin{example} The \verb!feedback! environment gives the student information after they enter an answer. Later, we expect to be able to tailor feedback to the student's answer.

\begin{verbatim}
\begin{question} Find a second-degree polynomial with leading coefficient $1$ that has a local minimum at $(3, -4)$.
\begin{prompt} $y = \answer{x^2 - 6x + 5}$ \end{prompt}
\begin{feedback}
The easiest way to accomplish this is to move the function $y=x^2$ to the right by $3$ and down by $4$.
\end{feedback}
\end{question}
\end{verbatim}

Produces:

\begin{question} Find a second-degree polynomial with leading coefficient $1$ that has a local minimum at $(3, -4)$.
\begin{prompt} $y = \answer{x^2 - 6x + 5}$ \end{prompt}
\begin{feedback}
The easiest way to accomplish this is to move the function $y=x^2$ to the right by $3$ and down by $4$.
\end{feedback}
\end{question}

\end{example}

\begin{example} 
Hints are implemented with the \verb!hint! environment, inside a \verb!question!, \verb!problem! or \verb!exercise!  environment. Online the hints are hidden behind a ``Reveal Hint'' button. In print, the hint is displayed below the question.

\begin{verbatim}
\begin{question}
  Solve for $x$: $x^2 = 25$. 
  \begin{prompt} 
    How many solutions did you get? $\answer{2}$ 
  \end{prompt}
  \begin{hint} 
    Did you forget $\pm$?
  \end{hint}
\end{question}
\end{verbatim}

Produces:

\begin{question}
  Solve for $x$: $x^2 = 25$. 
  \begin{prompt} 
    How many solutions did you get? $\answer{2}$ 
  \end{prompt}
  \begin{hint} 
    Did you forget $\pm$?
  \end{hint}
\end{question}

\end{example}

\section{Theorems} \label{TheoremEnvironments}

\begin{example} Structural statements that build up topics like ``Theorem'' appear in green, and are numbered both online and in print. These include Theorem, Axiom, Conjecture, Corollary, Proposition, Lemma, Claim, Condition, Idea, Definition, Conclusion, and Summary.

\begin{verbatim}
\begin{theorem} 
  Every cubic polynomial has a real root. 
\end{theorem}
\end{verbatim}

Produces:

\begin{theorem} 
  Every cubic polynomial has a real root. 
\end{theorem}
\end{example}

\begin{example} Remarks and observations like ``example'' appear in yellow, and are numbered both online and in print. In particular, sub-examples like the one below will be renumbered. The supported environments are: Example, Observation, Fact, Remark, Algorithm, Notation, and Criterion.

\begin{verbatim}
\begin{example}
  This example has a sub-numbering system since it is inside another example.
\end{example}
\end{verbatim}

Produces:

\begin{example}
  This example has a sub-numbering system since it is inside another example.
\end{example}
\end{example}

\begin{example} Warnings appear in red and are numbered online and in print. These include Warning and Paradox.

\begin{verbatim}
\begin{paradox}
  To walk one mile, you must first walk one-half mile, then one-quarter mile, then one-eighth mile, then one-sixteenth mile, and so on forever. This means you must complete infinitely many tasks to finish walking one mile. How can you possibly do this in a finite amount of time?
\end{paradox}
\end{verbatim}

Produces:

\begin{paradox}
  To walk one mile, you must first walk one-half mile, then one-quarter mile, then one-eighth mile, then one-sixteenth mile, and so on forever. This means you must complete infinitely many tasks to finish walking one mile. How can you possibly do this in a finite amount of time?
\end{paradox}
\end{example}

\section{Embedded Content} \label{EmbeddedContent}

\begin{example} A simple HTML link uses the \verb!link! command. In print, the URL is displayed in a footnote.

\begin{verbatim}
For more information, look at \link[a Cheese Catalog]{http://www.cheese.com}.
\end{verbatim}

Produces:

For more information, look at \link[a Cheese Catalog]{http://www.cheese.com}.
\end{example}

\begin{example} Youtube videos can be embedded using just their code (rather than the entire URL) using the \verb!youtube! command.

\begin{verbatim}
\youtube{dQw4w9WgXcQ}
\end{verbatim}

Produces:

\youtube{dQw4w9WgXcQ}
\end{example}

\begin{example} GeoGebra worksheets hosted at GeoGebra Tube can be embedded in the same way, using the \verb!geogebra! command. The \verb!geogebra! command takes three arguments: the ID of the worksheet, the width in pixels, and the height in pixels. In print, this command only displays a link to the geogebra worksheet.

\begin{verbatim}
\geogebra{222093}{711}{697}
\end{verbatim}

Produces: 

\geogebra{222093}{711}{697}

\begin{remark} 
  Currently, all optional arguments to the embedded Geogebra worksheet (like Show Menubar) default to false.
\end{remark}

\end{example}

\begin{example} If you have some HTML content that simply should not do anything in print, and you really want to include it, you can use the \verb!HTML! environment. 

\begin{verbatim}
\begin{html}
    <blink>Blinking text</blink> 
\end{html}
\end{verbatim}

Produces: 

??

\end{example}


\section{Code} \label{CodeAnswers}

\begin{example}
If you are displaying computer code to the student, you can use the \verb!code! or \verb!python! environments.

\begin{verbatim}
\begin{code}
This is in code environment.
    This is indented on the second line.

This is on the fourth line.
\end{code}
\end{verbatim}

Produces:

\begin{code}
This is in code environment.
    This is indented on the second line.

This is on the fourth line.
\end{code}
\end{example}

\begin{example}
If you are asking your students to write python code, for example trying to get them to type \verb!|return x**2!, you can use the \verb!python! environment and some validation functions.

\begin{verbatim}
\begin{question}
Model the function $f(x) = x^2$ as a python function. Try using \verb|return x**2|
\begin{python}
  def f(x):
  return #your code here

  def validator():
  return (f(4) == 16) and (f(-5) == 25)
\end{python}
\end{question}
\end{verbatim}

Produces:

\begin{question}
Model the function $f(x) = x^2$ as a python function. Try using \verb|return x**2|
\begin{python}
  def f(x):
  return #your code here

  def validator():
  return (f(4) == 16) and (f(-5) == 25)
\end{python}
\end{question}
\end{example}




\section{Free-Response Answer Type} \label{FRAnswerType}

\begin{example}The free response environment gives students access to a \LaTeX\ editor. 

\begin{verbatim}
\begin{question}
  Tell us a little about yourself!
  \begin{freeResponse}
      My name is Professor X. I am 42 years old, originally from Columbus, OH. The most recent 
      math I learned was functional analysis, three years ago. My favorite ice cream is Superman. 
      My favorite function is $f(x) = e^{-x^2}$.
  \end{freeResponse}
\end{question}
\end{verbatim}

Produces:

\begin{question}
  Tell us a little about yourself!
  \begin{freeResponse}
      My name is Professor X. I am 42 years old, originally from Columbus, OH. The most recent 
      math I learned was functional analysis, three years ago. My favorite ice cream is Superman. 
      My favorite function is $f(x) = e^{-x^2}.$
  \end{freeResponse}
\end{question}

\begin{remark}
Clicking on \verb!View model solution! shows the user
whatever you typed in the  \verb!freeResponse! environment.
\end{remark}
\end{example}

\section{Miscellaneous Structure} \label{StructuralEnvironments}

\begin{example}
The \verb!example! and \verb!explanation! environments serve as containers with nice labels. 

\begin{verbatim}
\begin{example}
The functions $f(x) = \frac{x}{1+x^3}$ and $g(x) = 0$ bound a region in the first quadrant that has finite area, but has no center of mass. 
\end{example}
\end{verbatim}

Produces:
\begin{example}
The functions $f(x) = \frac{x}{1+x^3}$ and $g(x) = 0$ bound a region in the first quadrant that has finite area, but has no center of mass. 
\end{example}
\end{example}

\section{Prompt} \label{Prompts}

\begin{example} If you would like to prompt a student online but not in your handouts, you can use the \verb!prompt! command. Prompts display on the web but do not display when the document class is called with a \verb!handout! class key. This allows you to demand certain formats for answers that will be checked online, while leaving more freedom for the answer format in person. The prompt command does nothing online.

\begin{verbatim}
\begin{question}
Find an equation for a line that has slope 5 and passes through the point (4, 6). 
\begin{prompt} 
  $y = \answer{5x - 14}$.
\end{prompt} 
\end{question}
\end{verbatim}

Produces:

\begin{question}
Find an equation for a line that has slope 5 and passes through the point (4, 6). 
\begin{prompt} 
  $y = \answer{5x - 14}$.
\end{prompt} 
\end{question}

\end{example}

\section{Dialogue} \label{Dialogue}

\begin{example} If you would like to display a dialogue between two fictional excited students, use the \verb!dialogue! environment and optional arguments on the items. 

\begin{verbatim}
\begin{dialogue}
  \item[Nathan] I think we should use the quadratic formula. What do you think?
  \item[Hans] I hate using the quadratic formula. It doesn't generalize to degree 5, which bothers me. Could we do something else?
\end{dialogue}
\end{verbatim}

Produces:

\begin{dialogue}
  \item[Nathan] I think we should use the quadratic formula. What do you think?
  \item[Hans] I hate using the quadratic formula. It doesn't generalize to degree 5, which bothers me. Could we do something else?
\end{dialogue}
\end{example}



\section{Metadata} \label{Metadata}

\begin{example} To include course outcomes in your questions, use the outcome command. In print, the outcomes will be placed in a footnote.

\begin{verbatim}
\outcome{Students should continue developing abstract thinking.}
\end{verbatim}

Produces:

\outcome{Students should continue developing abstract thinking.}
\end{example}

\end{document}
